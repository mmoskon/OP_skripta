\chapter{Okolje Jupyter}

\section{Interaktivni zvezki}
Okolje Jupyter predstavlja alternativo okolju IDLE, ki pa uporablja tudi nekoliko drugačen zapis programov. Programe namreč zapisujemo v tako imenovane \emph{interaktivne zvezke} \angl{IPython Notebooks}, s končnico \texttt{ipynb}. Preden si okolje podrobneje pogledamo, ga moramo namestiti. Spet uporabimo orodje \texttt{pip}:
\begin{lstlisting}[language=bash]
> pip install jupyter
\end{lstlisting}
Zdaj lahko okolje jupyter zaženemo, tako da se v ukazni vrstici našega operacijskega sistema premaknemo v mapo, kjer imamo shranjene datoteke, s katerimi bomo delali, in zaženemo ukaz:
\begin{lstlisting}[language=bash]
> jupyter notebook
\end{lstlisting}
S tem smo pognali strežnik okolja Jupyter \angl{Jupyter server}, s katerim se povežemo preko spletnega brskalnika, ki se po izvedbi zgornjega ukaza prav tako avtomatsko zažene. 

\section{Celice, tipi celic in njihovo poganjanje}
Najbolje je, da delovanje okolja Jupyter poskusimo kar na živem zgledu. Zgled s plačami, ki smo ga naredili v prejšnjem poglavju, je v obliki Juypter zvezka na voljo na \href{https://raw.githubusercontent.com/mmoskon/OP_skripta/master/resitve/place.ipynb}{povezavi}\footnote{\url{https://raw.githubusercontent.com/mmoskon/OP_skripta/master/resitve/place.ipynb}}. Prenesimo ga na svoj računalnik in shranimo v mapo, iz katere smo pognali Juypter. Zdaj bi morali datoteko z imenom \texttt{place.ipynb} videti v začetnem oknu okolja Jupyter. Odprimo jo s klikom nanjo. Vidimo, da je zvezek sestavljen iz dveh tipov celic. 

Prvi tip celic nudi razlago. Zapisane so v jeziku \texttt{Markdown}, ki predstavlja relativno preprost \emph{označevalni jezik} oziroma jezik za oblikovanje besedila. Če posamezno celico dvakrat poklikamo, lahko vidimo njeno izvorno kodo. Če hočemo celico spet pretvoriti v končno obliko, jo poženemo. To lahko naredimo s kombinacijo tipk \texttt{Ctrl + Enter} (poženi celico) oziroma \texttt{Shift + Enter} (poženi celico in skoči na naslednjo).

Drugi tip celic vsebuje kodo v jeziku Python. Te celice so označene z oznako \texttt{In}. Poganjamo jih na enak način kot celice tipa \texttt{Markdown}. Ko določeno celico poženemo, se pod njo pojavi njena izhodna (\texttt{Out}) celica, ki prikazuje rezultat njene izvedbe. Celice si med seboj delijo imenski prostor, kar pomeni, da lahko do spremenljivk, ki smo jih definirali v posamezni celici, dostopamo tudi iz ostalih celic. 

V zvezke lahko dodajamo nove celice, celice brišemo, kopiramo in spreminjamo tipe. Podrobneje v razlago okolja Jupyter ne bomo šli, saj je zelo intuitivno za uporabo, zahteva pa nekaj vaje. Tako kot zahteva vajo in trening tudi programiranje samo. Lotimo se ga...


%jedro IPython \angl{IPython kernel}, ki bo izvajalo naše programe