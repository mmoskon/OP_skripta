\chapter{Okolje Jupyter}

\section{Interaktivni zvezki}
Okolje Jupyter predstavlja alternativo okolju IDLE, ki pa uporablja tudi nekoliko drugačen zapis programov. Programe namreč zapisujemo v tako imenovane \emph{interaktivne zvezke} \angl{IPython Notebooks}, s končnico \texttt{ipynb}. Preden si okolje podrobneje pogledamo, ga moramo namestiti. Spet uporabimo orodje \texttt{pip}:
\begin{lstlisting}[language=bash]
> pip install jupyter
\end{lstlisting}
Zdaj lahko okolje jupyter zaženemo, tako da se v ukazni vrstici našega operacijskega sistema premaknemo v mapo, kjer imamo shranjene datoteke, s katerimi bomo delali, in zaženemo ukaz:
\begin{lstlisting}[language=bash]
> jupyter notebook
\end{lstlisting}
S tem smo pognali strežnik okolja Jupyter \angl{Jupyter server}, s katerim se povežemo preko spletnega brskalnika, ki se po izvedbi zgornjega ukaza prav tako zažene. 

\section{Celice, tipi celic in njihovo poganjanje}
Najbolje je, da delovanje okolja Jupyter poskusimo kar na živem zgledu. 



jedro IPython \angl{IPython kernel}, ki bo izvajalo naše programe