\chapter{Vizualizacija podatkov}

\section{Knjižnica Matplotlib in njena namestitev}
Za vizualizacijo podatkov bomo uporabljali knjižnico Matplotlib, ki predstavlja osnovo za risanje kakršnihkoli grafov. Ker v osnovni različici jezika Python še ni nameščen, ga moramo pred uporabo namestiti (če imate nameščeno distribucijo Anaconda, imate to knjižnico že nameščeno). Kot smo spoznali v poglavju \ref{ch:moduli} lahko za namestitev knjižnice uporabimo orodje \texttt{pip}, tako da zaženemo ukazno vrstico svojega operacijskega sistema (ne okolja IDLE) in vanjo vpišemo
\begin{lstlisting}[language=bash]
> pip install matplotlib
\end{lstlisting}
Podrobnejša navodila za namestitev paketov smo podali že v poglavju \ref{ch:moduli}, zato jih tu ne bomo podvajali. Če ima vaš računalnik povezavo z internetom, bo orodje \texttt{pip} samo preneslo potrebne namestitvene datoteke in knjižnico namestilo.

Za vizualizacija naših podatkov bomo uporabili Matplotlibov vmesnik \angl{interface} \texttt{pyplot}, ki nam risanje nekoliko olajša. V svoje programe ga bomo uvozili takole:
\begin{lstlisting}[language=Python, showstringspaces=false]
import matplotlib.pyplot as plt
\end{lstlisting}
Zdaj lahko do funkcij za risanje grafov dostopamo takole:
\begin{lstlisting}[language=Python, showstringspaces=false]
plt.ime_funkcije(argumenti)
\end{lstlisting}

\texttt{Funkciji \texttt{plot} in \texttt{show}}
