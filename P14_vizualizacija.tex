\chapter{Vizualizacija podatkov}

\section{Knjižnica Matplotlib in njena namestitev}
Za vizualizacijo podatkov bomo uporabljali knjižnico Matplotlib, ki predstavlja osnovo za risanje kakršnihkoli grafov. Ker v osnovni različici jezika Python še ni nameščen, ga moramo pred uporabo namestiti (če imate nameščeno distribucijo Anaconda, imate to knjižnico že nameščeno). Kot smo spoznali v poglavju \ref{ch:moduli} lahko za namestitev knjižnice uporabimo orodje \texttt{pip}, tako da zaženemo ukazno vrstico svojega operacijskega sistema (ne okolja IDLE) in vanjo vpišemo
\begin{lstlisting}[language=bash]
> pip install matplotlib
\end{lstlisting}
Podrobnejša navodila za namestitev paketov smo podali že v poglavju \ref{ch:moduli}, zato jih tu ne bomo podvajali. Če ima vaš računalnik povezavo z internetom, bo orodje \texttt{pip} samo preneslo potrebne namestitvene datoteke in knjižnico namestilo.

Za vizualizacija naših podatkov bomo uporabili Matplotlibov vmesnik \angl{interface} \texttt{pyplot}, ki nam risanje precej olajša. V svoje programe ga bomo uvozili takole:
\begin{lstlisting}[language=Python, showstringspaces=false]
import matplotlib.pyplot as plt
\end{lstlisting}
Zdaj lahko do funkcij za risanje grafov dostopamo takole:
\begin{lstlisting}[language=Python, showstringspaces=false]
plt.ime_funkcije(argumenti)
\end{lstlisting}

\section{Funkciji \texttt{plot} in \texttt{show}}

Začeli bomo s funkcijo \texttt{plot}, ki omogoča izris črtnega grafa \angl{line plot}. V osnovi ji lahko podamo zgolj en seznam. Poskusimo:
\begin{lstlisting}[language=Python, showstringspaces=false]
>>> Y = [1, 3, 9, 12]
>>> plt.plot(Y)
[<matplotlib.lines.Line2D object at 0x000001DD7FB28860>]
\end{lstlisting}
Nekaj se je očitno zgodilo, grafa pa še vedno ne vidimo. Funkcija \texttt{plot} namreč deluje tako, da grafe riše v ozadju in te dodaja na risalno površino, ki pa jo pokaže, šele ko pokličemo funkcijo \texttt{show}.
\begin{lstlisting}[language=Python, showstringspaces=false]
>>> plt.show()
\end{lstlisting}
Zdaj se je pokazal graf, ki ga prikazuje slika \ref{img:plt1}.
\begin{figure}
    \includegraphics[width=\linewidth]{img/plt1.png}
    \caption{Črtni graf, pri čemer smo podali koordinate točk na osi $y$.}
    \label{img:plt1}
\end{figure}
Točke, ki smo jih podali funkciji \texttt{plot}, očitno predstavljajo koordinate $y$ narisanega grafa. Točke na osi $x$ je Matplotlib določil kar glede na indekse točke v seznamu \texttt{Y}. To lahko spremenimo, tako da funkcijo \texttt{plot} pokličemo z dvema seznamoma, pri čemer prvi določa koordinate točk na osi $x$ drugi pa koordinate točk na osi $y$. Seznama morata biti seveda enakih dolžin. Do enakega grafa kot zgoraj, bi lahko prišli takole: 
\begin{lstlisting}[language=Python, showstringspaces=false]
>>> Y = [1, 3, 9, 12]
>>> X = range(len(Y))
>>> plt.plot(X, Y)
[<matplotlib.lines.Line2D object at 0x000002042BEBC128>]
plt.show()
\end{lstlisting}
Os $x$ bi lahko tudi spremenili. Poskusimo:
\begin{lstlisting}[language=Python, showstringspaces=false]
>>> X = [1,3,4,5]
>>> Y = [1, 3, 9, 12]
>>> plt.plot(X,Y)
[<matplotlib.lines.Line2D object at 0x000002042A9D0DA0>]
>>> plt.show()
\end{lstlisting}
Graf, ki smo ga narisali tako, prikazuje slika \ref{img:plt2}.
\begin{figure}
    \includegraphics[width=\linewidth]{img/plt2.png}
    \caption{Črtni graf, pri čemer smo podali koordinate točk na obeh oseh.}
    \label{img:plt2}
\end{figure}
Na sliki je prikazan samo zadnji graf. Kam je izginil prejšnji? Kot smo že omenili knjižnica Matplotlib grafe riše na risalni površini v ozadju. Te prikaže, ko pokličemo funkcijo \texttt{show}, hkrati pa takrat risalno površino tudi počisti. Če hočemo na isti sliki prikazati več grafov, bomo pred klicanjem funkcije \texttt{show} pač narisali več grafov. Poskusimo to kar na zgledu s plačami, in sicer bi radi narisali podatke o povprečnih bruto plačah. Predpostavljali bomo, da smo funkcijo \texttt{uvozi\_place} shranili v program \texttt{place\_beri.py} in da se ta program nahaja v naši trenutni delovni mapi. Najprej bomo uvozili funkcijo za uvoz podatkov o plačah zraven pa še knjižnico Matplotlib:
\begin{lstlisting}[language=Python, showstringspaces=false]
>>> from place_beri import uvozi_place
>>> import matplotlib.pyplot as plt
\end{lstlisting}
Potem preberimo podatke o plačah:
\begin{lstlisting}[language=Python]
>>> place = uvozi_place('place.csv')
\end{lstlisting}
in potegnimo sezname is slovarjev:
\begin{lstlisting}[language=Python]
>>> MJ = place['Javni sektor']['mesec']
>>> ZJ = place['Javni sektor']['bruto']
>>> MZ = place['Zasebni sektor']['mesec']
>>> ZZ = place['Zasebni sektor']['bruto']
\end{lstlisting}
Zdaj bomo kot koordinate na osi $x$ podali podatke o mesecih, kot koordinate na osi $y$ pa podatke o zneskih in poklicali funkcijo za prikaz grafa. Takole:
\begin{lstlisting}[language=Python, showstringspaces=false]
>>> plt.plot(MJ, ZJ)
>>> plt.plot(MZ, ZZ)
>>> plt.show()
\end{lstlisting}
Več grafov lahko na isto sliko narišemo tudi tako, da naštejemo pare seznamov kar po vrsti. Takole: 
\begin{lstlisting}[language=Python, showstringspaces=false]
>>> plt.plot(MJ, ZJ, MZ, ZZ)
>>> plt.show()
\end{lstlisting}
Kljub temu, da je rezultat v zgornjih dveh primerih enak, bomo raje uporabljali prvi način.
Zapišimo zdaj vse skupaj kot program \texttt{risi\_place.py}.
\begin{lstlisting}[language=Python, showstringspaces=false,numbers=left]
from place_beri import uvozi_place # funkcija za uvoz
import matplotlib.pyplot as plt 
MJ, ZJ, MZ, ZZ = uvozi_place('place.csv')
plt.plot(MJ, ZJ) # riši javni sektor
plt.plot(MZ, ZZ) # riši zasebni sektor
plt.show() # prikaži graf
\end{lstlisting}
Rezultat izvedbe zgornjega programa prikazuje slika \ref{img:plt3}.
\begin{figure}
    \includegraphics[width=\linewidth]{img/plt3.png}
    \caption{Osnovni izris podatkov o plačah.}
    \label{img:plt3}
\end{figure}

\section{Dodajanje oznak}
Vsak graf seveda potrebuje oznake. Označiti moramo kaj prikazuje posamezna os, za kar lahko uporabimo funkciji \texttt{xlabel} in \texttt{ylabel}, ki kot argument prejmeta niz, ki ga želimo prikazati. V našem primeru bi bilo smiselno napisati takole:
\begin{lstlisting}[language=Python, showstringspaces=false]
plt.xlabel("mesec")
plt.ylabel("znesek [EUR]")
\end{lstlisting}
Dodamo lahko tudi naslov grafa z uporabo funkcije \texttt{title}. Takole:
\begin{lstlisting}[language=Python, showstringspaces=false]
plt.title("Povprečne mesečne plače")
\end{lstlisting}

Manjka seveda tudi legenda. Kaj prikazuje modra linija in kaj oranžna? Legendo lahko dodamo tako, da oznake dodamo še posameznemu grafu kar med izrisom. Funkciji \texttt{plot} lahko preko opcijskega argumenta \texttt{label} podamo niz, ki predstavlja oznako grafa, ki ga bo izrisala. V našem primeru bi to naredili takole:
\begin{lstlisting}[language=Python, showstringspaces=false]
plt.plot(MJ, ZJ, label='Javni sektor') # riši javni sektor
plt.plot(MZ, ZZ, label='Zasebni sektor') # riši zasebni sektor
\end{lstlisting}
Če želimo legendo pokazati, moramo poklicati še funkcijo \texttt{legend}, ki prikaz legende vklopi:
\begin{lstlisting}[language=Python, showstringspaces=false]
plt.legend()
\end{lstlisting}
Vsebino legende, ki jo želimo izpisati bi lahko podali tudi neposredno funkciji \texttt{legend}. Takole:
\begin{lstlisting}[language=Python, showstringspaces=false]
plt.legend(['Javni sektor', 'Zasebni sektor'])
\end{lstlisting}
Pri tem moramo paziti na to, da oznake v legendi podajamo v enakem vrstnem redu, kot smo izvajali risanje grafov.


Zapišimo celoten program, ga poženimo in poglejmo rezultat. 
\begin{lstlisting}[language=Python, showstringspaces=false,numbers=left]
from place_beri import uvozi_place # funkcija za uvoz
import matplotlib.pyplot as plt

# uvozi podatke v slovar
place = uvozi_place('place.csv')
# pridobi sezname iz slovarja
MJ = place['Javni sektor']['mesec']
ZJ = place['Javni sektor']['bruto']
MZ = place['Zasebni sektor']['mesec']
ZZ = place['Zasebni sektor']['bruto']

plt.plot(MJ, ZJ) # riši javni sektor
plt.plot(MZ, ZZ) # riši zasebni sektor

# dodaj oznake
plt.xlabel("mesec")
plt.ylabel("znesek [EUR]")
plt.title("Povprečne mesečne plače")
plt.legend(['Javni sektor', 'Zasebni sektor'])

plt.show() # prikaži graf
\end{lstlisting}
Rezultat izvedbe zgornjega programa prikazuje slika \ref{img:plt4}.
\begin{figure}
    \includegraphics[width=\linewidth]{img/plt4.png}
    \caption{Izris podatkov o plačah z dodanimi oznakami.}
    \label{img:plt4}
\end{figure}

\section{Še malo prilagajanja oznak}
Kar nam še vedno ni všeč na sliki \ref{img:plt4} je neberljiv izpis na osi $x$. Če graf približamo, vidimo, da je na oseh izpisan podatek o mesecu. Mogoče bi bilo bolje, če bi ta podatek izpisali samo vsak januar, poleg tega pa bi informacijo o mesecu izpustili. Poskusimo odrezati rezino po mesecih od začetka do konca, pri čemer za korak nastavimo vrednost 12.
\begin{lstlisting}[language=Python, showstringspaces=false]
>>> MJ[::12]
['2014M01', '2015M01', '2016M01', '2017M01', '2018M01',
'2019M01', '2020M01']
\end{lstlisting}
Pripravimo si seznam \texttt{oznake}, ki bo vseboval samo podatke o letih. Vzeli bomo vsak 12-ti podatek iz obstoječega seznama mesecev (izhajamo lahko bodisi is seznama \texttt{MJ} ali \texttt{MZ}), pri čemer bomo upoštevali samo prve štiri znake (podatek o letu). 
\begin{lstlisting}[language=Python, showstringspaces=false]
oznake = []
for mj in MJ[::12]: # vzamemo vsako 12-to oznako
    oznake.append(mj[:4]) # vzamemo samo podatek o letu
\end{lstlisting}
Določiti moramo še lokacije, kjer bomo te oznake prikazali. Trenutno so oznake prikazane na lokacijah, ki se ujemajo z njihovimi indeksi, torej bi lahko lokacije oznak dobili s seznamoma \texttt{range(len(MJ))} ter \texttt{range(len(MZ))}. Ker bi radi prikazali vsako 12-to oznako, bomo morali torej upoštevati tudi vsako 12-to lokacijo. Takole:
\begin{lstlisting}[language=Python, showstringspaces=false]
# vsaka 12-ta lokacija
lokacije = range(0, len(MJ), 12) 
\end{lstlisting}

Lokacijo in vsebino oznak lahko zdaj našemu risarju podamo preko funkcije \texttt{xticks} (če bi želeli prilagajati oznake na osi $y$, bi uporabili funkcijo \texttt{yticks}):
\begin{lstlisting}[language=Python, showstringspaces=false]
plt.xticks(lokacije, oznake)
\end{lstlisting}

Celoten program je zdaj sledeč:
\begin{lstlisting}[language=Python, showstringspaces=false,numbers=left]
from place_beri import uvozi_place # funkcija za uvoz
import matplotlib.pyplot as plt

# uvozi podatke v slovar
place = uvozi_place('place.csv')
# pridobi sezname iz slovarja
MJ = place['Javni sektor']['mesec']
ZJ = place['Javni sektor']['bruto']
MZ = place['Zasebni sektor']['mesec']
ZZ = place['Zasebni sektor']['bruto']


# dodaj oznake
plt.xlabel("mesec")
plt.ylabel("znesek [EUR]")
plt.title("Povprečne mesečne plače")
plt.legend(['Javni sektor', 'Zasebni sektor'])

oznake = []
for mj in MJ[::12]: # vzamemo vsako 12-to oznako
    oznake.append(mj[:4]) # vzamemo samo podatek o letu

# vsaka 12-ta lokacija
lokacije = range(0, len(MJ), 12) 

plt.xticks(lokacije, oznake)

plt.show() # prikaži graf
\end{lstlisting}
Rezultat izvedbe programa prikazuje slika \ref{img:plt5}.
\begin{figure}
    \includegraphics[width=\linewidth]{img/plt5.png}
    \caption{Izris podatkov o plačah s prilagojenimi oznakami na osi $x$.}
    \label{img:plt5}
\end{figure}

Kaj pa če bi imeli v podatkih o plačah še kakšen sektor več? Ker imamo podatke shranjene v dokaj prilagodljivi strukturi (slovarju), bi lahko izris naredili neodvisno od števila sektorjev. Enostavno se sprehodimo čez ključe slovarja in rišemo. Takole:
\begin{lstlisting}[language=Python]
for sektor in place:
    mesec = place[sektor]['mesec']
    znesek = place[sektor]['neto']
    plt.plot(mesec, znesek, label=sektor)
\end{lstlisting}
Tokrat smo oznake grafov dodajali že kar med izrisovanjem preko argumenta \texttt{label}. Pri risanju oznak na osi $x$ lahko uporabimo kar spremenljivko \texttt{mesec}, v kateri so ostali podatki zadnjega sektorja (po sprehodu z zanko \texttt{for}). Celotna koda je sledeča: 
\begin{lstlisting}[language=Python, showstringspaces=false,numbers=left]
from place_beri import uvozi_place # funkcija za uvoz
import matplotlib.pyplot as plt

# uvozi podatke v slovar
place = uvozi_place('place.csv')
# pridobi sezname iz slovarja
MJ = place['Javni sektor']['mesec']
ZJ = place['Javni sektor']['bruto']
MZ = place['Zasebni sektor']['mesec']
ZZ = place['Zasebni sektor']['bruto']


plt.plot(MJ, ZJ) # riši javni sektor
plt.plot(MZ, ZZ) # riši zasebni sektor

# dodaj oznake
plt.xlabel("mesec")
plt.ylabel("znesek [EUR]")
plt.title("Povprečne mesečne plače")
plt.legend(['Javni sektor', 'Zasebni sektor'])

oznake = []
for mj in MJ[::12]: # vzamemo vsako 12-to oznako
    oznake.append(mj[:4]) # vzamemo samo podatek o letu

# vsaka 12-ta lokacija
lokacije = range(0, len(MJ), 12) 

# prikaži oznake na osi x
plt.xticks(lokacije, oznake)

plt.show() # prikaži graf
\end{lstlisting}

\section{Ostale prilagoditve izrisa}
Če nam grafi še vedno niso všeč, se lahko igramo naprej. Preko funkcije \texttt{plot} lahko nastavljamo barvo izrisa (argument \texttt{color}), debelino črte (argument \texttt{linewidth}), tip črte (argument \texttt{linestyle}) in še marsikaj. Poleg tega lahko določamo razpon osi (funkcija \texttt{axis}), rišemo več podgrafov (funkcija \texttt{subplot}) in graf shranjujemo v datoteko (funkcija \texttt{savefig}). Možnosti je res veliko in jih tukaj ne bomo več naštevali. Primere različnih grafov, ki jih lahko izrišemo z uporabo knjižnice Matplotlib, si lahko bralec pogleda (in prosto dostopno kodo prilagodi za risanje svojih grafov) na povezavi \url{https://matplotlib.org/gallery}.

\section{Ostali tipi grafov}

Matplotlib poleg črtnega diagrama (funkcija \texttt{plot}) omogoča risanje tudi ostalih tipov grafov, npr. stolpičnega diagrama \angl{bar plot} s funkcijo \texttt{bar}, histograma s funkcijo \texttt{hist}, kvartilnega diagrama s funkcijo \texttt{box} itd. Poglejmo si še primer izrisa stolpičnega diagrama, pri čemer bomo prikazali podatke o povprečnih plačah za leto 2018. Najprej iz podatkov izluščimo zgolj podatke za leto 2018. Hkrati se bomo morali sprehajati čez mesece in zneske. Ker sta seznama poravnana (isti indeks se nanaša na isti mesec), lahko naredimo sprehod s pomočjo funkcije \texttt{zip}. Znotraj sprehoda pogledamo, če se mesec nanaša na leto 2018 in v tem primeru mesec in znesek dodamo v nova seznama, ki se nanašata na leto 2018. To naredimo za javni in zasebni sektor posebej:
\begin{lstlisting}[language=Python, showstringspaces=false]
# uvozi podatke v slovar
place = uvozi_place('place.csv')
# pridobi sezname iz slovarja
MJ = place['Javni sektor']['mesec']
ZJ = place['Javni sektor']['bruto']
MZ = place['Zasebni sektor']['mesec']
ZZ = place['Zasebni sektor']['bruto']

MJ_2018 = []
ZJ_2018 = []
MZ_2018 = []
ZZ_2018 = []

for mj, zj in zip(MJ, ZJ):
    if "2018" in mj:
        MJ_2018.append(mj)
        ZJ_2018.append(zj)

for mz, zz in zip(MZ, ZZ):
    if "2018" in mz:
        MZ_2018.append(mz)
        ZZ_2018.append(zz)
\end{lstlisting}

Zdaj lahko podatke narišemo, pri čemer bomo za izris uporabili funkcijo \texttt{bar}. Ena izmed razlik med funkcijo \texttt{bar} in \texttt{plot} je, da moramo pri prvi lokacije stolpcev na osi $x$ vedno podati. Uporabimo lahko kar funkcijo \texttt{range}:
\begin{lstlisting}[language=Python, showstringspaces=false]
plt.bar(range(len(ZJ_2018)), ZJ_2018)
plt.bar(range(len(ZZ_2018)), ZZ_2018)
\end{lstlisting}

Z uporabo funkcije \texttt{xticks} lahko določimo še oznake na osi $x$:
\begin{lstlisting}[language=Python, showstringspaces=false]
plt.xticks(range(len(MJ)), MJ)
\end{lstlisting}

Poglejmo si rezultat
\begin{lstlisting}[language=Python, showstringspaces=false]
plt.show()
\end{lstlisting}
Prikazuje ga slika \ref{img:plt6}.
\begin{figure}
    \includegraphics[width=\linewidth]{img/plt6.png}
    \caption{Izris podatkov o plačah za leto 2018 s stolpičnim diagramom.}
    \label{img:plt6}
\end{figure}
Oznake na osi $x$ so zopet moteči. Grafu bi lahko dodali naslov, da gre za leto 2018, na osi $x$ pa prikazali samo informacijo o mesecu. Tako bo postalo vse skupaj nekoliko bolj pregledno. Najprej iz oznak odstranimo podatek o letu. To lahko naredimo že med filtiranjem podatkov za leto 2018, kjer namesto stavkov \texttt{MJ\_2018.append(mj)} in \texttt{MZ\_2018.append(mz)} uporabimo stavka \texttt{MJ\_2018.append(mj[-3:])} in \texttt{MZ\_2018.append(mz[-3:])}. Naslov grafa dodamo s sledečo vrstico:
\begin{lstlisting}[language=Python, showstringspaces=false]
plt.title('Podatki o plačah za leto 2018')
\end{lstlisting}
Dodajmo še legendo:
\begin{lstlisting}[language=Python, showstringspaces=false]
plt.legend(['Javni sektor', 'Zasebni sektor'])
\end{lstlisting}

Celoten program je zdaj sledeč:
\begin{lstlisting}[language=Python, showstringspaces=false,numbers=left]
from place_beri import uvozi_place # funkcija za uvoz
import matplotlib.pyplot as plt

# uvozi podatke v slovar
place = uvozi_place('place.csv')
# pridobi sezname iz slovarja
MJ = place['Javni sektor']['mesec']
ZJ = place['Javni sektor']['bruto']
MZ = place['Zasebni sektor']['mesec']
ZZ = place['Zasebni sektor']['bruto']

# izluščimo podatke za leto 2018
MJ_2018 = []
ZJ_2018 = []
MZ_2018 = []
ZZ_2018 = []

for mj, zj in zip(MJ, ZJ):
    if "2018" in mj:
        MJ_2018.append(mj[-3:]) # brez leta
        ZJ_2018.append(zj)

for mz, zz in zip(MZ, ZZ):
    if "2018" in mz:
        MZ_2018.append(mz[-3:]) # brez leta
        ZZ_2018.append(zz)


plt.bar(range(len(ZJ_2018)), ZJ_2018)
plt.bar(range(len(ZZ_2018)), ZZ_2018)

plt.xticks(range(len(MJ_2018)), MJ_2018)

plt.title('Podatki o plačah za leto 2018')
plt.legend(['Javni sektor', 'Zasebni sektor'])
plt.show()
\end{lstlisting}
Rezultat izvedbe programa prikazuje slika \ref{img:plt7}.
\begin{figure}
    \includegraphics[width=\linewidth]{img/plt7.png}
    \caption{Dopolnjen in popravljen izris podatkov o plačah za leto 2018 s stolpičnim diagramom.}
    \label{img:plt7}
\end{figure}



\section{Risanje matematičnih funkcij}
