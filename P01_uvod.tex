\chapter{Uvod}

\section{Računalništvo}

Računalništvo je v zadnjih letih postalo nepogrešljiv del našega vsakdana. Računalniki niso zgolj naprave, s katerimi igramo igrice, pišemo seminarske naloge in brskamo po internetu, temveč predstavljajo računalniki ključne komponente praktično vseh sodobnih elektronskih ali z elektroniko podprtih naprav od telefonov pa vse do pametnih kuhinjskih aparatov in avtomobilov. Eden od pomembnih dejavnikov, ki je povzročil takšen razmah računalnikov, je zagotovo internet, ki je za novejše generacije zemljanov postal nepogrešljiv učno-izobraževalni, prostočasni in tudi socialni pripomoček. 

Računalnike torej srečujemo na vsakem koraku.  Prav je, da vsaj približno razumemo, kako računalniki razmišljajo. Če ne drugega, to v današnjem času spada v kontekst splošne razgledanosti. Poleg tega nam razumevanje delovanja računalniških algoritmov odpira tudi malo bolj kritičen pogled do sodobnih tehnologij. 

%Primer ene izmed povsod prisotnih aplikacij računalništva je umetna inteligenca, ki hote ali nehote posega v našo zasebnost in preko t.i. priporočilnih sistemov prilagaja zadetke, ki jih dobimo pri iskanju na platformah, kot je Google, izbira oglase, ki bodo prikazani med našim brskanjem po internetu in sortira objave, ki jih vidimo najprej na socialnih omrežjih ter tako (lahko) ustvarja iluzijo sveta, ki nas obdaja. Teorije zarote pravijo, da so se v preteklosti z uporabo manipulacij takih sistemov npr. dobivale predsedniške volitve. Naloga nas kot izobražencev je, da znamo oceniti, ali so take špekulacije upravičene in da se znamo do njih opredeliti in jih kritično ovrednotiti. Dobro je torej, če vsaj približno razumemo kako stvari, ki jih dnevno uporabljamo, delujejo. In to, da se zavedamo, da umetna inteligenca ni čisto nič drugega, kot zbirka različnih matematičnih pristopov, ki je v široko uporabo prešla prav zdaj, ko imamo na voljo veliko količino podatkov (podatki o nas se zbirajo praktično na vsakem našem koraku) in veliko računsko moč (računalniki so v današnjem času dovolj močni, da lahko relativno hitro obdelajo zelo velike količine zajetih podatkov). Dobro je tudi to, da vemo, da lahko odločanje, kaj določenemu uporabniku prikaže t.i. priporočilni sistem, `nekdo` zlahka zmanipulira, če je motivacija za to dovolj velika.

\section{Zakaj se učiti programiranja}

Dobro je torej vedeti, kako računalniki razmišljajo. Računalniki pa sami po sebi ne razmišljajo, ampak namesto njih `razmišljajo` programi, ki jih računalniki poganjajo. In programiranje je veja računalništva, ki se ukvarja s pisanjem računalniških programov. Če znamo programirati, torej vsaj približno vemo, na kakšen način računalniki razmišljajo.

Programiranje je poleg tega nadvse uporabno, saj nam omogoča, da pišemo lastne programe in da naloge, ki bi zahtevale, da razmišljamo mi, predamo računalnikom. Obstaja rek, da so vsi dobri programerji veliki lenivci. Obstaja pa tudi prepričanje, da bi moralo učenje programiranja spadati v osnovno izobrazbo vsakega posameznika, saj nas uči algoritmičnega načina razmišljanja. In da bi se morali otroci konceptov programiranja učiti že v prvih letih šolanja, če ne že v vrtcih.

Algoritmičen način razmišljanja se osredotoča na reševanje problemov na tak način, da določimo recept oziroma algoritem, ki opisuje korake, s katerimi bomo problem čimbolj učinkovito rešili. Tak način razmišljanja se zna izkazati za nadvse koristnega pri vsakodnevnem premagovanju problemov. In določanje algoritma predstavlja osnovo pri programiranju. Kompleksen (ali pa tudi ne) problem želimo v tem kontekstu razbiti na sekvenco korakov oziroma sestavin, ki jih bomo uporabili pri reševanju. Te na koncu med seboj povežemo v smiselno zaporedje, to zamešamo v naš program in problem je rešen. Programiranje velikokrat spominja na kuhanje, ko poskušamo določiti recept izbrani jedi, potem pa ta recept prevedemo v mešanje sestavin, njihovo obdelavo z ustreznimi postopki (kuhanje, vzhajanje, peka), včasih pa seveda tudi malo improviziramo. In tako nekako bomo tudi programirali. 

\section{Računalništvo in kemija (in druge vede)}

Ne samo da je računalništvo postalo del našega vsakdana, ampak je postalo tudi podporna veda za različna strokovna področja, med katerimi je tudi kemija. Po eni strani računalništvo našemu strokovnemu delu daje podporo in povečuje našo učinkovitost, saj nudi številna orodja za zajem, shranjevanje, obdelavo, vizualizacijo in deljenje podatkov, omogoča krmiljenje kompleksnih naprav itd. 

Po drugi strani obstajajo številne aplikacije, ki jih brez računalništva sploh ne bi bilo. Kot prvi primer vzemimo Nobelovo nagrado za kemijo iz leta 2013, ki je bila podeljena za računalniške modele in simulacije v kemiji. Tovrstni modeli odpirajo nova odkritja na področju razumevanja sistemov, ki jih je težko analizirati zgolj eksperimentalno. Drugi primer je uporaba računalnikov pri analizi velike količine eksperimentalnih podatkov, ki jih ročno ne bi mogli nikoli pregledati, kaj šele, da bi iz njih potegnili kaj uporabnega. V tem kontekstu lahko omenimo večletni projekt sekvenciranja in anotacije človeškega genoma, ki je temeljil na podlagi sekvenciranja in sestavljanja fragmentov genoma različnih ljudi. Pri tem je bil projekt zelo ambiciozen tudi iz računalniške perspektive. Poskrbeti je bilo namreč treba za shranjevanje ogromne količine podatkov in napisati algoritme za odkrivanje vzorcev v podatkih ter združevanje fragmentov v končno celoto. Povezati je bilo potrebno najmodernejše sekvenatorje z najmodernejšimi računalniki oziroma superračunalniki in mediji za shranjevanje velike količine podatkov. Kljub temu, da rezultati uspešno izvedenega projekta niso pripeljali do tako revolucionarnih odkritij, kot so nekateri sprva domnevali (npr. uspešno zdravljenje vseh genetskih bolezni), je v sled uspešno zaključenem projektu sekvenciranje genoma posameznika v današnjem času postalo skoraj rutinska klinična metoda za diagnostiko in ocenjevanje tveganja za razvoj določenih bolezni. Določene države so celo investirale sredstva za določitev svojega nacionalnega genoma (na Islandiji so npr. vsaj delno sekvencirali genom več kot polovici prebivalcem) in usmeritev financiranja raziskav v smeri zdravljenja bolezni, za katere so njihovi prebivalci bolj dovzetni. Poleg tega je cenovno dostopno sekvenciranje odprlo nove veje pri razvoju medicine, kot je npr. sistemska in personalizirana medicina, ki na podlagi genetskih lastnosti posameznika(ce) poskuša določiti optimalno terapijo posebej zanj(o).

\section{Kaj je programiranje?}

Računalnik lahko v grobem razdelimo na dva dela, in sicer na strojno opremo, ki predstavlja `platformo`, na kateri pač poganjamo programe in ki je brez programske opreme `mrtva`, in programsko opremo, ki predstavlja `živčni sistem` računalnika. %To lahko v grobem naprej delimo na `sistemsko programsko opremo`, ki predstavlja osnovo za delo z računalnikom in med drugim omogoča komunikacijo med človekom in računalnikom in `uporabniško programsko opremo`, ki predstavlja uporabne programe, ki jih pač ljudje uporabljamo pri svojem delu. 

Programiranje se ukvarja s pisanjem programov oziroma razvojem programske opreme. Proces izvajanja naročil računalniku, tj. naredi nekaj zame, zakomplicira dejstvo, da mi računalnika v osnovi ne razumemo, računalnik pa v osnovi ne razume nas. Programiranje predstavlja proces zapisovanja naših navodil v jeziku, ki je pogojno razumljiv nam (če pač znamo programirati), hkrati pa je razumljiv posebnim programom (ali pa skupini programov), ki naša zaporedja ukazov oziroma kodo, prevedejo \angl{compile} ali pretolmačijo \angl{intepret} v zapis, ki je razumljiv računalniku. 
Sekvenco navodil/ukazov oziroma kodo pa lahko pišemo v različnih programskih jezikih. Eni izmed teh so bliže računalnikovem osnovnem jeziku (nižjenivojski jeziki) in nam omogočajo več svobode, imajo pa zato tudi več prostora za to, da lahko naredimo kakšno neumnost ter so na splošno težji za uporabo. V to skupino bi lahko uvrstili npr. programski jezik C. Drugi so bližje našemu načinu razmišljanja (višjenivojski jeziki), zato je programiranje z njimi lažje, je pa izvedba tako napisanih programov mogoče nekoliko počasnejša. V to skupino bi lahko uvrstili npr. programski jezik Python.

\section{Zakaj Python?}

Ta knjiga opisuje programiranje v jeziku Python, ki je po mnenju marsikoga daleč najprimernejši jezik za začetek učenja programiranja. Koda napisana v tem jeziku je enostavno berljiva in intuitivna za razumevanje. Poleg tega programiranje v jeziku Python poteka na nekoliko višjem nivoju (pravimo, da je nivo abstrakcije višji), kar pomeni, da nam ni potrebno zelo natančno vedeti, kaj se npr. v ozadju dogaja s pomnilniškim prostorom, zato lahko s programiranjem brez dodatnega teoretiziranja začnemo kar takoj. Enostavnost jezika pa ne omejuje njegove uporabnosti, Python se namreč uporablja pri razvoju številnih popularnih aplikacij (oziroma programov), kot je npr. brskalnik Google, YouTube, DropBox in Instagram. V letu 2020 je bil tretji najbolj popularen programski jezik (za programskima jezikoma Java in C), njegova popularnost pa še narašča (medtem ko jezikoma Java in C upada). 

Poleg tega, da je jezik Python hitro učljiv, to še ne pomeni, da ni zmogljiv. Njegova dodatna prednost je razpoložljivost številnih `knjižnic`,\footnote{Knjižnica predstavlja zbirko že napisanih delov kode, ki jo lahko enostavno pokličemo iz našega programa. Če bi naredili analogijo s kuhanjem, bi lahko rekli, da imamo z uporabo knjižnic na razpolago bolj kompleksne sestavine, ki jih samo še sestavimo skupaj -- npr. namesto, da pečemo testo za torto od začetka, vzamemo že narejen biskvit, ki ga samo še namažemo in okrasimo.} ki razširjajo njegove osnovne funkcionalnosti. Na voljo imamo tudi številne knjižnice, ki jih lahko uporabljamo pri reševanju problemov npr. na področju kemije, kot je knjižnica \texttt{chempy} za uporabo v analizni kemiji, knjižnica \texttt{chemlab} namenjeno računski kemiji ter vizualizaciji molekul in knjižnica \texttt{biopython} za izvajanje bioinformatičnih analiz.

Kaj pa počasnost? Hitrost izvajanja programov v jeziku Python je sicer počasnejša od hitrosti programov, ki so napisani v jezikih, kot je C. Vseeno pa lahko pri pisanju programov uporabimo knjižnice, ki `računsko intenzivne` operacije izvajajo hitreje, ali pa `računsko intenzivne` dele naših programov napišemo v jeziku, kot je npr. jezik C, in te povežemo s preostalo kodo napisano v jeziku Python.

Osnov programiranja v jeziku Python se boste lahko torej naučili relativno hitro, poleg tega pa ga boste lahko uporabljali tudi, če se boste po zaključku učenja osnov lotili bolj resnega programiranja. 

\section{Kaj me čaka na koncu knjige?} 

Ta knjiga je namenjena učenju osnov programiranja v jeziku Python. Pogledali si bomo osnovne gradnike, ki jih uporabljamo pri programiranju in par pogosto uporabljenih knjižnic, predvsem za analizo in vizualizacijo podatkov. Ko boste knjigo prebrali, naj bi znali brez problemov sledeče:
\begin{itemize}
    \item pisati računalniške programe za zajem podatkov (od uporabnika, iz datotek, iz spleta ipd.),
    \item pisati računalniške programe za obdelavo podatkov (določitev najpomembnejših podatkov, sortiranje podatkov, izvajanje statističnih analiz podatkov),
    \item pristopiti k vizualizaciji podatkov,
    \item pristopiti k reševanju problemov na algoritmičen način.
\end{itemize}
Nenazadnje boste dobili osnovno znanje programiranja, ki ga boste v prihodnosti zlahka nadgradili, če boste tako hoteli.

Za konec uvoda pa še opozorilo. Programiranja se še nihče ni naučil z branjem knjig. Kdor hoče znati programirati, mora programirati. Knjiga služi zgolj kot pripomoček, ki ga lahko uporabljate pri učenju, njeno branje pa je zagotovo potrebno dopolniti s treningom. Zglede v knjigi torej poskusite v čimvečjem obsegu rešiti sami, rešujte vaje in poskusite znanje programiranja uporabiti pri reševanju problemov, s katerimi se soočate pri drugih predmetih (namesto npr. uporabe orodja Excel). Programiranje je namreč kot šport in za spretnost zahteva svoj trening. Trenirajte.

\section{Nekaj navodil za branje knjige}
Posamezno poglavje v knjigi vsebuje zglede in njihove rešitve. Opis teh je podan v poševnem tekstu, poleg tega pa je začetek zgledov in rešitev označen in oštevilčen. Konec posamezne rešitve označuje simbol $\triangle$. Koda v jeziku Python je zapisana v pisavi \texttt{Courier}. Če so na začetku vrstice pred kodo zapisani trije lomljeni oklepaji oziroma trije znaki za relacijo večje (\texttt{>}\texttt{>}\texttt{>}), to pomeni, da kodo zapisujemo v ukazno vrstico okolja Python. Če so vrstice kode oštevilčene, to pomeni, da kodo pišemo v obliki programa. 

