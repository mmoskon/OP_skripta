\chapter{Oblikovanje nizov}

\section{Delo z nizi}

Nize smo do zdaj že dodobra spoznali. Preden se lotimo dela z datotekami pa si moramo pogledati še nekaj metod, ki jih lahko nad nizi uporabimo.

Ponovimo najprej osnovne operacije, ki smo jih do zdaj že izvajali nad nizi. Nad nizi lahko izvajamo zanko \texttt{for}, pri čemer v vsaki iteraciji zanke, dostopamo do enega znaka niza. Takole:
\begin{lstlisting}[language=Python]
>>> niz = "abeceda"
>>> for znak in niz:
	print(znak)
a
b
e
c
e
d
\end{lstlisting}
Nize lahko med sabo seštevamo, čemur smo rekli lepljenje ali konkatenacija, poleg tega pa jih lahko množimo s celimi števili:
\begin{lstlisting}[language=Python]
>>> "abc" + " " + "def"
'abc def'
>>> "abc" * 3
'abcabcabc'
\end{lstlisting}
Njihove elemente lahko tudi indeksiramo in delamo rezine. Ne moremo pa jih spreminjati. Odstranjevanje podniza z besedico \texttt{del} torej ne bo šlo skozi:
\begin{lstlisting}[language=Python]
>>> niz = "abeceda"
>>> niz[2:5]
'ece'
>>> del niz[2:5]
TypeError: 'str' object does not support item deletion
\end{lstlisting}
Zakaj ne? Ker so nizi nespremenljivi. Alternativa je seveda ta, da naredimo nov niz, ki odraža željeno spremembo. V zgornjem primeru bi torej lahko naredili nov niz z lepljenjem dveh rezin. Eno pred podnizom, ki ga želimo odstraniti, in drugo za podnizom, ki ga želimo odstraniti. Takole:
\begin{lstlisting}[language=Python]
>>> niz = "abeceda"
>>> niz[:2] + niz[5:]
'abda'
\end{lstlisting}
Iz niza smo tako odstranili podniz \texttt{"ece"}. Če želimo, da se sprememba odraža tudi preko spremenljivke z imenom \texttt{niz}, bomo nov (spremenjen) niz priredili temu imenu:
\begin{lstlisting}[language=Python]
>>> niz = "abeceda"
>>> niz = niz[:2] + niz[5:]
>>> niz
'abda'
\end{lstlisting}
Zapakirajmo vse skupaj v funkcijo.
\begin{zgled}
Napiši funkcijo \texttt{odstrani}, ki sprejme dva niza in iz prvega niza odstrani prvo pojavitev drugega niza ter spremenjen niz vrne. Če se drugi podniz v prvem ne pojavi, naj funkcija vrne nespremenjen niz.
\end{zgled}

\begin{resitev}
V funkciji moramo najprej preveriti, če in kje se v nizu nahaja podniz. To lahko naredimo tako, da se sprehajamo od začetka do konca niza in režemo rezine dolžine podniza. Da lahko delamo rezine, se moramo seveda sprehajati po indeksih niza. To bi izgledalo nekako takole:
\begin{lstlisting}[language=Python]
>>> niz = "abeceda"
>>> podniz = "ece"
>>> for i in range(len(niz)):
	niz[i:i+len(podniz)]
'abe'
'bec'
'ece'
'ced'
'eda'
'da'
'a'
\end{lstlisting}
Znotraj zanke bomo preverjali, če je odrezana rezina enaka podnizu (zgoraj je to bilo v 3. iteraciji zanke). V tem primeru ga bomo odstranili (podobno kot prej) in vrnili rezultat. Napišimo celotno funkcijo.
\begin{lstlisting}[language=Python,numbers=left]
def odstrani(niz, podniz):
    for i in range(len(niz)):
        rezina = niz[i:i+len(podniz)]
        if rezina == podniz:
            zacetek = i # začetek rezanja
            konec = i + len(podniz) # konec rezanja
            return niz[:zacetek] + niz[konec:]
    return niz # nismo našli - vrni nespremenjen niz
\end{lstlisting}
Vprašanje je še ali mora funkcija res vračati spremenjen niz ali je dovolj, če niz, ki ga funkcija sprejme kot argument, spreminjamo znotraj funkcije (kot smo to delali pri seznamih, slovarjih in množicah). Spremenjen niz moramo seveda eksplicitno vrniti, saj je niz nespremenljiv. Spremembe, ki jih vršimo nad imenom, za katerim se skriva niz, se izven funkcije ne bodo ohranile, zato je potrebno nov (spremenjen) niz vrniti s stavkom \texttt{return}. 
\end{resitev}

Tole je bilo še kar zakomplicirano. Izkaže pa se, da nam te in podobnih funkcij ni treba pisati, saj večinoma že obstajajo. Najdemo jih med metodami nizov. Izbrane (tj. tiste, ki jih bomo uporabljali pogosteje) si bomo podrobneje pogledali v nadaljevanju poglavja.

\section{Iskanje podnizov}
V prejšnjem zgledu se lahko uporabi rezin deloma izognemo z uporabo metode \texttt{startswith}, ki vrne vrednost \texttt{True}, če se niz, preko katerega smo metodo poklicali začne s podanim podnizom. Rešitev bi torej lahko nekoliko poenostavili:
\begin{lstlisting}[language=Python,numbers=left]
def odstrani(niz, podniz):
    for i in range(len(niz)):
        if niz[i:].startswith(podniz):
            zacetek = i # začetek rezanja
            konec = i + len(podniz) # konec rezanja
            return niz[:zacetek] + niz[konec:]
    return niz # nismo našli - vrni nespremenjen niz
\end{lstlisting}
Poleg metode \texttt{startswith} lahko pri iskanju uporabimo še metodo \texttt{endswith}, ki preverja, če se niz konča s podanim podnizom, metodo \texttt{count}, ki prešteje število pojavitev podniza in metodo \texttt{find}, ki vrne lokacijo podniza oziroma --1, če podniza v nizu ni. Povadimo še uporabo metode \texttt{find}:
\begin{lstlisting}[language=Python,numbers=left]
def odstrani(niz, podniz):
    i = niz.find(podniz)
    if i >= 0: # ali je podniz v nizu
        zacetek = i # začetek rezanja
        konec = i + len(podniz) # konec rezanja
        return niz[:zacetek] + niz[konec:]
    return niz
\end{lstlisting}

\section{Odstranjevanje in spreminjanje (pod)nizov}
Podnize pa lahko odstranimo zgolj s klicem metode \texttt{replace}, ki vse pojavitve prvega argumenta zamenja z drugim argumentom. Poskusimo
\begin{lstlisting}[language=Python]
>>> niz = "abeceda"
>>> niz.replace("a","e")
'ebecede'
\end{lstlisting}
Metoda podanega niza ne spreminja. Seveda, saj ga ne more, ker so nizi nespremenljivi. Metoda torej niza ne more spremeniti, lahko pa vrne nov niz, ki odraža spremembo. Če želimo s klicem metode spremeniti vrednost, ki stoji za imenom spremenljivke, bomo rezultat klica priredili spremenljivki. Takole:
\begin{lstlisting}[language=Python]
>>> niz = "abeceda"
>>> niz = niz.replace("a","e")
>>> niz
'ebecede'
\end{lstlisting}
Kako lahko uporabimo metodo pri odstranjevanju podnizov? Tako, da kot drugi argument podamo prazen niz.
\begin{lstlisting}[language=Python]
>>> niz = "abeceda"
>>> niz.replace("a","")
'beced'
\end{lstlisting}
V navodilu prejšnje naloge je bilo zahtevno, da odstranimo samo prvo pojavitev podniza. Tudi to lahko rešimo z dodatnim argumentom, ki ga metoda \texttt{replace} sprejema. Poglejmo si izpis funkcije \texttt{help}:
\begin{lstlisting}[language=Python]
>>> help("".replace)
Help on built-in function replace:

replace(old, new, count=-1, /) method of builtins.str instance
    Return a copy with all occurrences of substring old replaced
    by new.
    
      count
        Maximum number of occurrences to replace.
        -1 (the default value) means replace all occurrences.
    
    If the optional argument count is given, only the first count 
    occurrences are replaced.
\end{lstlisting}
Podamo lahko tudi tretji argument \texttt{count}, ki pove koliko zamenjav naj metoda naredi (privzeto naredi vse zamenjave). Zdaj lahko našo funkcijo poenostavimo do konca:
\begin{lstlisting}[language=Python,numbers=left]
def odstrani(niz, podniz):
    return niz.replace(podniz, "", 1)
\end{lstlisting}

Pogosto uporabljeni metodi za spreminjanje nizov sta še metodi \texttt{upper} in \texttt{lower}, ki spremenita vse črke niza v velike oziroma male. Ker Python loči med velikimi in malimi črkami, lahko ti dve metodi uporabimo, kadar te ločitve ne želimo imeti in vse pač spremenimo bodisi v velike bodisi v male črke. Povadimo na naslednjem zgledu.

\begin{zgled}
Napiši funkcijo \texttt{palindrom}, ki kot argument sprejme niz in vrne \texttt{True}, če je niz palindrom in \texttt{False}, če ni. Pri tem naj funkcija ne loči med malimi in velikimi črkami.
\end{zgled}

\begin{resitev}
Kako preveriti, če je niz palindrom, že znamo. Tokrat bomo dodali še to, da preverjanje ne loči med malimi in velikimi črkami. To bomo naredili tako, da bomo vse črke v nizu pred preverjanjem pretvorili v velike (ali pa v male).
\begin{lstlisting}[language=Python,numbers=left]
def palindrom(niz):
    niz = niz.upper()
    return niz == niz[::-1]
\end{lstlisting}
\end{resitev}

\section{Razdruževanje in združevanje nizov}
Funkcija \texttt{input} uporabnikov vnos vedno vrne zapisan kot niz. Prav tako bomo vsebino datotek (v naslednjem poglavju) vedno dobili najprej zapisano kot niz. V primeru, da uporabnik vnese eno število, lahko število zapisano kot niz enostavno pretvorimo preko funkcij \texttt{int} oziroma \texttt{float}. V primeru, da uporabnik naenkrat vnese več števil ali pa je vnos sestavljen iz števil in drugih znakov, ki jih mogoče želimo pretvoriti v kaj drugega pa ta pristop ni več mogoč. Deloma nas sicer lahko reši funkcija \texttt{eval}, katere uporaba zaradi varnosti v splošnem ni priporočljiva, poleg tega pa tudi ta slej ko prej odpove. V tem primeru moramo niz razčleniti \angl{parse} sami. Za ta namen lahko uporabimo metodo \texttt{split}, ki ji kot argument podamo ločilo \angl{separator}, preko katerega naj niz loči v seznam nizov. Poglejmo si primer:
\begin{lstlisting}[language=Python]
>>> niz = "jabolka,hruške,slive"
>>> niz.split(",")
['jabolka', 'hruške', 'slive']
\end{lstlisting}
Povadimo še na zgledu:

\begin{zgled}
Napiši program, ki na podlagi uporabnikovega vnosa izračuna in izpiše vsoto podanih števil. Uporabnik bo števila vnesel v eni vrstici in jih med seboj ločil s podpičji. 
\end{zgled}
\begin{resitev}
Program bo uporabnikov vnos (niz) ločil po podpičjih. Sledil bo sprehod čez dobljeni seznam, pretvorba nizov v števila in seštevanje.
\begin{lstlisting}[language=Python,numbers=left]
vnos = input("Vnesi števila ločena s podpičji: ")
seznam = vnos.split(";")
s = 0 # začetna vsota
for st in seznam:
    s += float(st) # pretvorba niza in prištevanje
print(s)
\end{lstlisting}
Zgled izvedbe programa:
\begin{lstlisting}[language=Python]
Vnesi števila ločena s podpičji: 20;40.5;60
120.5
\end{lstlisting}
\end{resitev}

Metodo \texttt{split} lahko pokličemo tudi brez podanega ločila, pri čemer bo privzeto za ločilo uporabljen \emph{prazen prostor} \angl{white space}. Prazen prostor predstavlja znake, ki na zaslonu predstavljajo točno to -- prazen prostor. Sem uvrstimo presledek, tabulator (\texttt{$\backslash$t}) in novo vrstico (\texttt{$\backslash$n}) (opomba: kombinacija znakov \texttt{$\backslash$t} in \texttt{$\backslash$n} pravzaprav predstavlja en sam znak). Dodatna prednost uporabe metode \texttt{split} brez argumenta je v tem, da bo več zaporednih ponovitev znakov za prazen prostor obravnaval kot eno samo. Poskusimo:
\begin{lstlisting}[language=Python]
>>> niz = "1    2 \n 3\t4"
>>> niz.split(" ")
['1', '', '', '', '2', '\n', '3\t4']
>>> niz.split()
['1', '2', '3', '4']
\end{lstlisting}
Klic metode z argumentom \texttt{" "} praznega prostora očitno ni odstranil v celoti, klic metode brez argumentov iz niza uspešno pobral samo relevantne vrednosti. 

Podobno, kot lahko podan niz glede na podano ločilo razbijemo na seznam nizov, lahko seznam nizov, glede na podano ločilo združimo v en niz. V tem primeru uporabimo metodo \texttt{join}, ki jo pokličemo nad nizom, ki predstavlja naše ločilo, kot argument pa ji podamo seznam nizov, ki ga želimo združiti. Takole: 
\begin{lstlisting}[language=Python]
>>> seznam = ["jabolka", "slive", "avokado"]
>>> ", ".join(seznam)
'jabolka, slive, avokado'
\end{lstlisting}


\section{Odstranjevanje \emph{praznega prostora}}

Dopolnimo najprej naše iskanje palindromov.
\begin{zgled}
Napiši funkcijo \texttt{palindrom}, ki kot argument sprejme niz in vrne \texttt{True}, če je niz palindrom in \texttt{False}, če ni. Pri tem naj funkcija ne loči med malimi in velikimi črkami, poleg tega pa naj ignorira presledke, znake za nove vrstice in tabulatorje.
\end{zgled}

\begin{resitev}
Želimo torej, da bi rešitev delovala tudi nad nečim takim kot je niz \texttt{"perica reže raci rep"}. Niz se prebere enako naprej kot nazaj, ampak samo v primeru, ko presledkov ne upoštevamo. Prva rešitev bo temeljila na tem, da vse bele prostore zamenjamo za prazne nize.
\begin{lstlisting}[language=Python,numbers=left]
def palindrom(niz):
    niz = niz.upper()
    niz = niz.replace(" ","")
    niz = niz.replace("\n","")
    niz = niz.replace("\t","")
    return niz == niz[::-1]
\end{lstlisting}
Alternativa je, da najprej nad nizom uporabimo metodo \texttt{split} brez podanega argumenta, ki bo niz ločila v seznam nizov, pri čemer bo odstranila ves prazen prostor. Zatem uporabimo metodo \texttt{join}, pri čemer kot ločilo uporabimo prazen niz. Takole:
\begin{lstlisting}[language=Python,numbers=left]
>>> niz = "perica \n reže \t raci rep"
>>> sez = niz.split()
>>> sez
['perica', 'reže', 'raci', 'rep']
>>> niz = "".join(sez)
>>> niz
'pericarežeracirep'
\end{lstlisting}
Druga rešitev je torej sledeča:
\begin{lstlisting}[language=Python,numbers=left]
def palindrom(niz):
    niz = niz.upper()
    sez = niz.split()
    niz = "".join(sez)
    return niz == niz[::-1]
\end{lstlisting}
Prve tri vrstice lahko združimo v eno:
\begin{lstlisting}[language=Python,numbers=left]
def palindrom(niz):
    niz = "".join(niz.upper().split())
    return niz == niz[::-1]
\end{lstlisting}
\end{resitev}

Beli prostor lahko torej odstranjujemo na različne načine. V določenih primerih pa želimo beli prostor odstraniti samo pred začetkom in po koncu \emph{prave} vsebine, vmes pa ne. V tem primeru lahko uporabimo metodo \texttt{strip}
\begin{lstlisting}[language=Python]
>>> niz = "\n   \t  Danes je lep dan!     \n"
>>> niz.strip()
'Danes je lep dan!'
\end{lstlisting}

%\section{Posebni znaki}

\section{Prilagajanje izpisa in formatiranje nizov}

Funkcija \texttt{print} podane argumente združi v niz, ki ga izpiše na zaslon. Pri tem med argumente vstavi presledke (\texttt{sep = " "}), na koncu izpisa pa gre v novo vrstico (\texttt{end = "$\backslash$n"}). Privzeto delovanje lahko spremenimo, tako da nastavimo (povozimo) privzete vrednosti izbirnima argumentom \texttt{sep} in \texttt{end}. 

Včasih pa to ni dovolj in bi bili radi z izpisovanjem nekoliko bolj ustvarjalni. V tem primeru lahko uporabimo metodo \texttt{format}. Metodo \texttt{format} pokličemo nad nizom, ki vsebuje t.~i. fiksne dele in dele, ki jih bo metoda zamenjala z argumenti, ki jih bomo podali ob klicu. Slednje znotraj niza označimo z zavitimi oklepaji (\texttt{\{} in \texttt{\}}). Osnovni primer uporabe je sledeč:
\begin{lstlisting}[language=Python]
>>> m = 1
>>> "{} meter/ov/a je {} centimeter/ov/a".format(m, m*100)
'1 meter/ov/a je 100 centimeter/ov/a'
\end{lstlisting}
Prvo pojavitev zavitega oklepaja je metoda \texttt{format} torej zamenjala s prvim argumentom, drugo pa z drugim. Znotraj zavitih oklepajev lahko tudi eksplicitno navedemo, kateri argument naj se uporabi pri zamenjavi. Takole:
\begin{lstlisting}[language=Python]
>>> m = 1
>>> "{0} m = {1} cm, torej je {1} cm = {0} m".format(m, m*100)
'1 m = 100 cm, torej je 100 cm = 1 m'
\end{lstlisting}
Do malo gršega izpisa pride, kadar imamo veliko decimalk:
\begin{lstlisting}[language=Python]
>>> cm = 12.673
>>> "{} cm = {} m".format(cm, cm/100)
'12.673 cm = 0.12673 m'
\end{lstlisting}
Rezultate lahko sicer zaokrožujemo z vgrajeno funkcijo \text{round}, ki ji kot argument podamo število decimalk. Alternativa je, da zaokroževanje podamo samo pri izpisu, tako da znotraj zavitih oklepajev malo bolj natančno povemo kako naj se izpis formatira. Takole:
\begin{lstlisting}[language=Python]
>>> cm = 12.673
>>> "{:5.1f} cm = {:5.2f} m".format(cm, cm/100)
' 12.7 cm =  0.13 m'
\end{lstlisting}
Kaj smo s tem povedali? Z dvopičjem povemo, da želimo izpis malo oblikovati. Število 5 podaja (najmanjše) število mest, ki naj jih izpis zasede. Iz izpisa je vidno, da je metoda format pred posamezno število vstavila presledek, saj je izpisano število dolgo 4 znake, mi pa smo povedali, da naj bo izpis dolg 5 znakov. Če bi bilo naše število daljše od 5 znakov, rezanja ne bi bilo, ampak bi \texttt{format} izpisal vse znake. Za piko smo podali število decimalk, ki naj bodo v izpisu. Pri centimetrih smo uporabili 1, pri metrih pa 2 decimalki. Oznaka \texttt{f} pomeni, da formatiramo decimalno število \angl{float}. Oblikovanje celih števil in nizov je enostavneje. V tem primeru podamo število mest, ki naj jih število (minimalno) zavzame. 
\begin{lstlisting}[language=Python]
>>> "{:3}: {:15}...{:5.2f}°C".format(1, "Ljubljana", 25.3)
'  1: Ljubljana      ...25.30°C'
\end{lstlisting}
Povemo lahko še ali naj se posamezen izpis poravna levo (\texttt{<}), desno (\texttt{>}) ali sredinsko (\texttt{\^}):
\begin{lstlisting}[language=Python]
>>> "{:<3}: {:<15}...{:>5.2f}°C".format(1, "Ljubljana", 25.3)
'1  : Ljubljana      ...25.30°C'
\end{lstlisting}
Presledke lahko zamenjamo tudi za kaj drugega, npr. za pike.
\begin{lstlisting}[language=Python]
>>> "{:<3}: {:.<15}...{:>5.2f}°C".format(1, "Ljubljana", 25.3)
'1  : Ljubljana.........25.30°C'
\end{lstlisting}
Zdaj lahko uporabo formatiranja demonstriramo na celotnem zgledu:
\begin{lstlisting}[language=Python]
>>> stevilke = [1,2,3]
>>> kraji = ["Ljubljana", "Maribor", "Nova Gorica"]
>>> temperature = [25.3, 21.32322, 26.433333]
>>> for st, kraj, temp in zip(stevilke, kraji, temperature):
	print("{:<3}: {:.<15}...{:>5.2f}°C".format(st, kraj, temp))
1  : Ljubljana.........25.30°C
2  : Maribor...........21.32°C
3  : Nova Gorica.......26.43°C
\end{lstlisting}

\section{Formatiranje nizov in \emph{f-Strings}}

Novejša in hitrejša alternativa metodi \texttt{format} je formatiranje nizov zapisanih v obliki, ki ji rečemo \emph{f-niz} oziroma \emph{f-Strings}. Tovrstne nize zapisujemo tako, da pred začetkom niza (pred navednice) zapišemo črko \texttt{f}. Python bo to razumel kot niz, ki ga mora še dodatno oblikovati. V osnovi tak niz zapišemo takole:
\begin{lstlisting}[language=Python]
>>> f'niz'
'niz'
\end{lstlisting}
V f-niz lahko v zavite oklepaje vstavimo spremenljivke, ki jih bo Python pri izpisu zamenjal za njihove vrednosti, podobno kot pri metodi \texttt{format}:
\begin{lstlisting}[language=Python]
>>> m = 1
>>> f"{m} meter/ov/a je {m*100} centimeter/ov/a"
'1 meter/ov/a je 100 centimeter/ov/a'
\end{lstlisting}
Tak način oblikovanja je nekoliko hitrejši kot oblikovanje z metodo \texttt{format}, predvsem pa je tak način oblikovanja bolj pregleden. Podobno kot pri metodi \texttt{format} lahko v zavite oklepaje podamo oblikovanje izpisa posamezne spremenljivke, le da je v tem primeru vrstni red malenkost drugačen, saj spremenljivko podamo kar v zavite oklepaje pred načinom njenega oblikovanja:
\begin{lstlisting}[language=Python]
f"{spremenljivka:oblikovanje}"
\end{lstlisting}
Poskusimo na zgledu:
\begin{lstlisting}[language=Python]
>>> cm = 12.673
>>> f"{cm:5.1f} cm = {cm/100:5.2f} m"
' 12.6 cm =  0.13 m'
\end{lstlisting}
Ostalo je zelo podobno oziroma enako kot pri uporabi metode \texttt{format}. Poskusimo še na zadnjem zgledu iz prejšnjega razdelka:
\begin{lstlisting}[language=Python]
>>> stevilke = [1,2,3]
>>> kraji = ["Ljubljana", "Maribor", "Nova Gorica"]
>>> temperature = [25.3, 21.32322, 26.433333]
>>> for st, kraj, temp in zip(stevilke, kraji, temperature):
	print(f"{st:<3}: {kraj:.<15}...{temp:>5.2f}°C")
1  : Ljubljana.........25.30°C
2  : Maribor...........21.32°C
3  : Nova Gorica.......26.43°C
\end{lstlisting}
Rezultat je torej enak kot v primeru uporabo metode \texttt{format}, je pa koda postala nekoliko krajša in bolj pregledna.